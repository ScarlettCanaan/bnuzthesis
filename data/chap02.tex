\chapter{中华人民共和国}
\label{cha:china}

\section{其它例子}
\label{sec:other}

在第~\ref{cha:intro} 章中我们学习了贝叶斯公式~(\ref{equ:chap1:bayes}),这里我们复
习一下:
\begin{equation}
\label{equ:chap2:bayes}
p(y|\mathbf{x}) = \frac{p(\mathbf{x},y)}{p(\mathbf{x})}=
\frac{p(\mathbf{x}|y)p(y)}{p(\mathbf{x})}
\end{equation}

\subsection{绘图}
\label{sec:draw}

本模板不再预先装载任何绘图包(如 \textsf{pstricks,pgf} 等),完全由用户来决定。
个人觉得 \textsf{pgf} 不错,不依赖于 Postscript。此外还有很多针对 \LaTeX{} 的
 GUI 作图工具,如 XFig(jFig), WinFig, Tpx, Ipe, Dia, Inkscape, LaTeXPiX,
jPicEdt, jaxdraw 等等。

\subsection{插图}
\label{sec:graphs}

强烈推荐《\LaTeXe\ 插图指南》!关于子图形的使用细节请参看 \textsf{subcaption} 宏包的说明文档。

\subsubsection{一个图形}
\label{sec:onefig}
一般图形都是处在浮动环境中。之所以称为浮动是指最终排版效果图形的位置不一定与源文
件中的位置对应\footnote{This is not a bug, but a feature of \LaTeX!},这也是刚使
用 \LaTeX{} 同学可能遇到的问题。如果要强制固定浮动图形的位置,请使用 \textsf{float} 宏包,
它提供了 \texttt{[H]} 参数,比如图~\ref{fig:xfig1}。
\begin{figure}[H] % use float package if you want it here
  \centering
  \includegraphics{thu-whole-logo}
  \caption{利用 Xfig 制图}
  \label{fig:xfig1}
\end{figure}

大学之道,在明明德,在亲民,在止于至善。知止而后有定;定而后能静;静而后能安;安
而后能虑;虑而后能得。物有本末,事有终始。知所先后,则近道矣。古之欲明明德于天
下者,先治其国;欲治其国者,先齐其家;欲齐其家者,先修其身;欲修其身者,先正其心;
欲正其心者,先诚其意;欲诚其意者,先致其知;致知在格物。物格而后知至;知至而后
意诚;意诚而后心正;心正而后身 修;身修而后家齐;家齐而后国治;国治而后天下
平。自天子以至于庶人,壹是皆以修身为本。其本乱而未治者 否矣。其所厚者薄,而其所
薄者厚,未之有也!

\hfill —— 《大学》


\subsubsection{多个图形}
\label{sec:multifig}

%如果多个图形相互独立,并不共用一个图形计数器,那么
%用 \texttt{minipage} 或者\texttt{parbox} 就可以。否则,请参看
%图~\ref{fig:big1-subfloat},它包含两个小图,分别是图~\ref{fig:subfig1}和
%图~\ref{fig:subfig2}。推荐使用\verb|subfloat|,因为可以像
%图~\ref{fig:big1-subfloat} 那样对齐子图的标题,也可以使
%用\textsf{subcaption}宏包的\verb|subcaption|(放在 minipage中,用法同\verb|caption|)。

%\begin{figure}[h]
  %\centering%
  %\subfloat{第一个小图形\label{fig:subfig1}}[3cm] %标题的长度,超过则会换行,如下一个小图。
    %{\includegraphics[height=3cm]{thu-fig-logo}}%
  %\hspace{4em}%
  %\subfloat{第二个小图形,注意这个图略矮些。如果标题很长的话,它会自动换行\label{fig:subfig2}}
      %{\includegraphics[height=2cm]{thu-text-logo}}
  %\caption{包含子图形的大图形(subfloat示例)}
  %\label{fig:big1-subfloat}
%\end{figure}
%\begin{figure}[h]
  %\centering%
  %\begin{subfigure}{3cm}
    %\includegraphics[height=3cm]{thu-fig-logo}
    %\caption{第一个小图形}
  %\end{subfigure}%
  %\hspace{4em}%
  %\begin{subfigure}{0.5\textwidth}
    %\includegraphics[height=2cm]{thu-text-logo}
    %\caption{第二个小图形,注意这个图略矮些。subfigure中同一行的子图在顶端对齐。}
  %\end{subfigure}
  %\caption{包含子图形的大图形(subfigure示例)}
  %\label{fig:big1-subfigure}
%\end{figure}

如果要把编号的两个图形并排,那么小页就非常有用了:
\begin{figure}
\begin{minipage}{0.48\textwidth}
  \centering
  \includegraphics[height=2cm]{thu-whole-logo}
  \caption{并排第一个图}
  \label{fig:parallel1}
\end{minipage}\hfill
\begin{minipage}{0.48\textwidth}
  \centering
  \includegraphics[height=2cm]{thu-whole-logo}
  \caption{并排第二个图}
  \label{fig:parallel2}
\end{minipage}
\end{figure}
