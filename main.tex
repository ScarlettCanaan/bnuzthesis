\documentclass[bachelor]{bnuthesis}
% 选项:
%   type=[bachelor|master|doctor|postdoctor], % 必选
%   secret,                                   % 可选
%   pifootnote,                               % 可选(建议打开)
%   openany|openright,                        % 可选,基本不用
%   arial,                                    % 可选,基本不用
%   arialtoc,                                 % 可选,基本不用
%   arialtitle                                % 可选,基本不用

% 所有其它可能用到的包都统一放到这里了,可以根据自己的实际添加或者删除。
\usepackage{bnutils}

% 定义所有的图片文件在 figures 子目录下
\graphicspath{{figures/}}

% 可以在这里修改配置文件中的定义。导言区可以使用中文。
% \def\myname{人生败犬琥珀}

\begin{document}

%%% 封面部分
\frontmatter
% !Mode:: "TeX:UTF-8"
% Author: Kohaku
%\secretlevel{绝密} \secretyear{10}

\ctitle{北京师范大学珠海分校学术论文\LaTeX{}模版}
%\makeatletter
\cdegree{学士}

\makeatother

\cauthor{琥珀} 
\csupervisor{XX \;教授}
%\cassosupervisor={某某教授}, % 副指导老师
%\ccosupervisor={某某某教授}, % 联合指导老师
\cdepartment[]{信息技术学院}
\cmajor{挖掘机修理}
\cnum{13XXXXXXXX}
  % 日期自动生成,若需指定按如下方式修改:
  % edate={December, 2005}
  %
  % 关键词用“英文逗号”分割
\cdate{2017年3月}

  % 日期自动使用当前时间,若需指定按如下方式修改:
  % cdate={超新星纪元},
  %
  % 博士后专有部分
  %cfirstdiscipline={计算机科学与技术},
  %cseconddiscipline={系统结构},
  %postdoctordate={2009年7月——2011年7月},
  %id={编号}, % 可以留空: id={},
  %udc={UDC}, % 可以留空
  %catalognumber={分类号}, % 可以留空

\etitle{An Introduction to \LaTeX{} Thesis Template of Beijing Normal University. Zhuhai v\version} 


  % 这块比较复杂,需要分情况讨论:
  % 1. 学术型硕士
  %    edegree:必须为Master of Arts或Master of Science(注意大小写)
  %             “哲学、文学、历史学、法学、教育学、艺术学门类,公共管理学科
  %              填写Master of Arts,其它填写Master of Science”
  %    emajor:“获得一级学科授权的学科填写一级学科名称,其它填写二级学科名称”
  % 2. 专业型硕士
  %    edegree:“填写专业学位英文名称全称”
  %    emajor:“工程硕士填写工程领域,其它专业学位不填写此项”
  % 3. 学术型博士
  %    edegree:Doctor of Philosophy(注意大小写)
  %    emajor:“获得一级学科授权的学科填写一级学科名称,其它填写二级学科名称”
  % 4. 专业型博士
  %    edegree:“填写专业学位英文名称全称”
  %    emajor:不填写此项

% 定义中英文摘要和关键字
\begin{cabstract}
  论文的摘要是对论文研究内容和成果的高度概括。摘要应对论文所研究的问题及其研究目
  的进行描述,对研究方法和过程进行简单介绍,对研究成果和所得结论进行概括。摘要应
  具有独立性和自明性,其内容应包含与论文全文同等量的主要信息。使读者即使不阅读全
  文,通过摘要就能了解论文的总体内容和主要成果。

  论文摘要的书写应力求精确、简明。切忌写成对论文书写内容进行提要的形式,尤其要避
  免“第 1 章……;第 2 章……;……”这种或类似的陈述方式。

  本文介绍京师范大学珠海分校论文模板 \bnuthesis{} 的使用方法。本模板符合学校的本科毕业生论文格式要求。

  本文的创新点主要有:
  \begin{itemize}
    \item 用例子来解释模板的使用方法;
    \item 用废话来填充无关紧要的部分;
    \item 一边学习摸索一边编写新代码。
  \end{itemize}

  关键词是为了文献标引工作、用以表示全文主要内容信息的单词或术语。关键词不超过 5
  个,每个关键词中间用分号分隔。(模板作者注:关键词分隔符不用考虑,模板会自动处
  理。英文关键词同理。)
\end{cabstract}

% 如果习惯关键字跟在摘要文字后面,可以用直接命令来设置,如下:
 \ckeywords{\TeX, \LaTeX, CJK, 模板, 论文}

\begin{eabstract}
   An abstract of a dissertation is a summary and extraction of research work
   and contributions. Included in an abstract should be description of research
   topic and research objective, brief introduction to methodology and research
   process, and summarization of conclusion and contributions of the
   research. An abstract should be characterized by independence and clarity and
   carry identical information with the dissertation. It should be such that the
   general idea and major contributions of the dissertation are conveyed without
   reading the dissertation.

   An abstract should be concise and to the point. It is a misunderstanding to
   make an abstract an outline of the dissertation and words ``the first
   chapter'', ``the second chapter'' and the like should be avoided in the
   abstract.

   Key words are terms used in a dissertation for indexing, reflecting core
   information of the dissertation. An abstract may contain a maximum of 5 key
   words, with semi-colons used in between to separate one another.
\end{eabstract}

 \ekeywords{\TeX, \LaTeX, CJK, template, thesis}

% 如果使用授权说明扫描页,将可选参数中指定为扫描得到的 PDF 文件名,例如:
% \makecover[scan-auth.pdf]
\makecover

%% 目录
\tableofcontents

%% 符号对照表
\input{data/denotation}


%%% 正文部分
\mainmatter
\chapter{带 English 的标题}
\label{cha:intro}

这是 \bnuthesis{} 的示例文档,基本上覆盖了模板中所有格式的设置。建议大家在使用模
板之前,除了阅读《\bnuthesis{}用户手册》,这个示例文档也最好能看一看。

小老鼠偷吃热凉粉;短长虫环绕矮高粱\footnote{韩愈(768-824),字退之,河南河阳(
  今河南孟县)人,自称郡望昌黎,世称韩昌黎。幼孤贫刻苦好学,德宗贞元八年进士。曾
  任监察御史,因上疏请免关中赋役,贬为阳山县令。后随宰相裴度平定淮西迁刑部侍郎,
  又因上表谏迎佛骨,贬潮州刺史。做过吏部侍郎,死谥文公,故世称韩吏部、韩文公。是
  唐代古文运动领袖,与柳宗元合称韩柳。诗力求险怪新奇,雄浑重气势。}。


\section{封面相关}
封面的例子请参看 cover.tex。主要符号表参看 denation.tex,附录和个人简历分别参看 appendix01.tex
和 resume.tex。里面的命令都很只管,一看即会\footnote{你说还是看不懂?怎么会呢?}。

\section{字体命令}
\label{sec:first}


{\kaishu 坡仙擅长行书、楷书,取法李邕、徐浩、颜真卿、杨凝式,而能自创新意。用笔丰腴
  跌宕,有天真烂漫之趣。与蔡襄、黄庭坚、米芾并称“宋四家”。能画竹,学文同,也喜
  作枯木怪石。论画主张“神似”,认为“论画以形似,见与儿童邻”;高度评价“诗中有
  画,画中有诗”的艺术造诣。诗文有《东坡七集》等。存世书迹有《答谢民师论文帖》、
  《祭黄几道文》、《前赤壁赋》、《黄州寒食诗帖》等。  画迹有《枯木怪石图》、《
  竹石图》等。}

{\fangsong 易与天地准,故能弥纶天地之道。仰以观於天文,俯以察於地理,是故知幽明之故。原
  始反终,故知死生之说。精气为物,游魂为变,是故知鬼神之情状。与天地相似,故不违。
  知周乎万物,而道济天下,故不过。旁行而不流,乐天知命,故不忧。安土敦乎仁,故
  能爱。范围天地之化而不过,曲成万物而不遗,通乎昼夜之道而知,故神无方而易无体。}

% 非本科生一般用不到幼圆与隶书字体。需要的同学请查看 ctex 文档。
{\ifcsname youyuan\endcsname\youyuan\else[无 \verb|youyuan| 字体。]\fi 有天地,然后
  万物生焉。盈天地之间者,唯万物,故受之以屯;屯者盈也,屯者物之始生也。物生必蒙,
  故受之以蒙;蒙者蒙也,物之穉也。物穉不可不养也,故受之以需;需者饮食之道也。饮
  食必有讼,故受之以讼。讼必有众起,故受之以师;师者众也。众必有所比,故受之以比;
  比者比也。比必有所畜也,故受之以小畜。物畜然后有礼,故受之以履。}

{\heiti 履而泰,然后安,故受之以泰;泰者通也。物不可以终通,故受之以否。物不可以终
  否,故受之以同人。与人同者,物必归焉,故受之以大有。有大者不可以盈,故受之以谦。
  有大而能谦,必豫,故受之以豫。豫必有随,故受之以随。以喜随人者,必有事,故受
  之以蛊;蛊者事也。}

{\ifcsname lishu\endcsname\lishu\else[无 \verb|lishu| 字体。]\fi 有事而后可大,故受
  之以临;临者大也。物大然后可观,故受之以观。可观而后有所合,故受之以噬嗑;嗑者
  合也。物不可以苟合而已,故受之以贲;贲者饰也。致饰然后亨,则尽矣,故受之以剥;
  剥者剥也。物不可以终尽,剥穷上反下,故受之以复。复则不妄矣,故受之以无妄。}

{\songti 有无妄然后可畜,故受之以大畜。物畜然后可养,故受之以颐;颐者养也。不养则不
  可动,故受之以大过。物不可以终过,故受之以坎;坎者陷也。陷必有所丽,故受之以
  离;离者丽也。}

\section{表格样本}
\label{chap1:sample:table} 

\subsection{基本表格}
\label{sec:basictable}
模板中关于表格的宏包有三个: \textsf{booktabs}、\textsf{array} 和 \textsf{longtabular},命令有一个 \verb|\hlinewd|。三线表可以用 \textsf{booktabs} 提供的 \verb|\toprule|、\verb|\midrule| 和 \verb|\bottomrule|。它们与 \textsf{longtable} 能很好的配合使用。如果表格比较简单的话可以直接用命令
\verb|hlinewd{xpt}|控制。
\begin{table}[htb]
  \centering
  \begin{minipage}[t]{0.8\linewidth} % 如果想在表格中使用脚注,minipage是个不错的办法
  \caption[模板文件]{模板文件。如果表格的标题很长,那么在表格索引中就会很不美
    观,所以要像 chapter 那样在前面用中括号写一个简短的标题。这个标题会出现在索
    引中。}
  \label{tab:template-files}
    \begin{tabularx}{\linewidth}{lX}
      \toprule[1.5pt]
      {\heiti 文件名} & {\heiti 描述} \\\midrule[1pt]
      thuthesis.ins & \LaTeX{} 安装文件,\textsc{DocStrip}\footnote{表格中的脚注} \\
      thuthesis.dtx & 所有的一切都在这里面\footnote{再来一个}。\\
      thuthesis.cls & 模板类文件。\\
      thuthesis.cfg & 模板配置文。cls 和 cfg 由前两个文件生成。\\
      thuthesis.bst    & 参考文献 BIB\TeX\ 样式文件。\\
      thuthesis.sty   & 常用的包和命令写在这里,减轻主文件的负担。\\
      \bottomrule[1.5pt]
    \end{tabularx}
  \end{minipage}
\end{table}

首先来看一个最简单的表格。表 \ref{tab:template-files} 列举了本模板主要文件及其功
能。请大家注意三线表中各条线对应的命令。这个例子还展示了如何在表格中正确使用脚注。
由于 \LaTeX{} 本身不支持在表格中使用 \verb|\footnote|,所以我们不得不将表格放在
小页中,而且最好将表格的宽度设置为小页的宽度,这样脚注看起来才更美观。

\subsection{复杂表格}
\label{sec:complicatedtable}

我们经常会在表格下方标注数据来源,或者对表格里面的条目进行解释。前面的脚注是一种
不错的方法,如果不喜欢脚注,可以在表格后面写注释,比如表~\ref{tab:tabexamp1}。
\begin{table}[htbp]
  \centering
  \caption{复杂表格示例 1}
  \label{tab:tabexamp1}
  \begin{minipage}[t]{0.8\textwidth} 
    \begin{tabularx}{\linewidth}{|l|X|X|X|X|}
      \hline
 \multirow{2}* & \multicolumn{2}{c|}{First Half} & \multicolumn{2}{c|}{Second Half}\\\cline{2-5}
      & 1st Qtr &2nd Qtr&3rd Qtr&4th Qtr \\ \hline
      East$^{*}$ &   20.4&   27.4&   90&     20.4 \\
      West$^{**}$ &   30.6 &   38.6 &   34.6 &  31.6 \\ \hline
    \end{tabularx}\\[2pt]
    \footnotesize 注:数据来源《\bnuthesis{} 使用手册》。\\
    *:东部\\
    **:西部
  \end{minipage}
\end{table}

此外,表~\ref{tab:tabexamp1} 同时还演示了另外一个功能:通过 \textsf{tabularx} 的
 \texttt{|X|} 扩展实现表格自动放大;
浮动体的并排放置一般有两种情况:1)二者没有关系,为两个独立的浮动体;2)二者隶属
于同一个浮动体。对表格来说并排表格既可以像图~\ref{tab:parallel1}、图~\ref{tab:parallel2} 
使用小页环境,也可以如图~\ref{tab:subtable} 使用子表格来做。图的例子参见第~\ref{sec:multifig} 节。

\begin{table}[htbp]
\noindent\begin{minipage}{0.5\textwidth}
\centering
\caption{第一个并排子表格}
\label{tab:parallel1}
\begin{tabular}{p{2cm}p{2cm}}
\toprule[1.5pt]
111 & 222 \\\midrule[1pt]
222 & 333 \\\bottomrule[1.5pt]
\end{tabular}
\end{minipage}%
\begin{minipage}{0.5\textwidth}
\centering
\caption{第二个并排子表格}
\label{tab:parallel2}
\begin{tabular}{p{2cm}p{2cm}}
\toprule[1.5pt]
111 & 222 \\\midrule[1pt]
222 & 333 \\\bottomrule[1.5pt]
\end{tabular}
\end{minipage}
\end{table}

\begin{table}[htbp]
\centering
\caption{并排子表格}
\label{tab:subtable}
\subfloat{第一个子表格}
{
\begin{tabular}{p{2cm}p{2cm}}
\toprule[1.5pt]
111 & 222 \\\midrule[1pt]
222 & 333 \\\bottomrule[1.5pt]
\end{tabular}
}
\hskip2cm
\subfloat{第二个子表格}
{
\begin{tabular}{p{2cm}p{2cm}}
\toprule[1.5pt]
111 & 222 \\\midrule[1pt]
222 & 333 \\\bottomrule[1.5pt]
\end{tabular}
}
\end{table}

不可否认 \LaTeX{} 的表格功能没有想象中的那么强大,不过只要足够认真,足够细致,
同样可以排出来非常复杂非常漂亮的表格。请参看表~\ref{tab:tabexamp2}。
\begin{table}[htbp]
  \centering\dawu[1.3]
  \caption{复杂表格示例 2}
  \label{tab:tabexamp2}
  \begin{tabular}[c]{|m{1.5cm}|c|c|c|c|c|c|}\hline
    \multicolumn{2}{|c|}{Network Topology} & \# of nodes & 
    \multicolumn{3}{c|}{\# of clients} & Server \\\hline
    GT-ITM & Waxman Transit-Stub & 600 &
    \multirow{2}{2em}{2\%}& 
    \multirow{2}{2em}{10\%}& 
    \multirow{2}{2em}{50\%}& 
    \multirow{2}{1.2in}{Max. Connectivity}\\\cline{1-3}
    \multicolumn{2}{|c|}{Inet-2.1} & 6000 & & & &\\\hline
    \multirow{2}{1.5cm}{Xue} & Rui  & Ni &\multicolumn{4}{c|}{\multirow{2}*{\bnuthesis}}\\\cline{2-3}
    & \multicolumn{2}{c|}{ABCDEF} &\multicolumn{4}{c|}{} \\\hline
\end{tabular}
\end{table}

要想用好论文模板还是得提前学习一些 \TeX/\LaTeX{}的相关知识,具备一些基本能力,掌
握一些常见技巧,否则一旦遇到问题还真是比较麻烦。我们见过很多这样的同学,一直以来
都是使用 Word 等字处理工具,以为 \LaTeX{}模板的用法也应该类似,所以就沿袭同样的思
路来对待这种所见非所得的排版工具,结果被折腾的焦头烂额,疲惫不堪。

如果您要排版的表格长度超过一页,那么推荐使用 \textsf{longtable} 或者 \textsf{supertabular}
宏包,模板对 \textsf{longtable} 进行了相应的设置,所以用起来可能简单一些。
表~\ref{tab:performance} 就是 \textsf{longtable} 的简单示例。
\begin{longtable}[c]{c*{6}{r}}
\caption{实验数据}\label{tab:performance}\\
\toprule[1.5pt]
 测试程序 & \multicolumn{1}{c}{正常运行} & \multicolumn{1}{c}{同步} & \multicolumn{1}{c}{检查点} & \multicolumn{1}{c}{卷回恢复}
& \multicolumn{1}{c}{进程迁移} & \multicolumn{1}{c}{检查点} \\
& \multicolumn{1}{c}{时间 (s)}& \multicolumn{1}{c}{时间 (s)}&
\multicolumn{1}{c}{时间 (s)}& \multicolumn{1}{c}{时间 (s)}& \multicolumn{1}{c}{
  时间 (s)}&  文件(KB)\\\midrule[1pt]
\endfirsthead
\multicolumn{7}{c}{续表~\thetable\hskip1em 实验数据}\\
\toprule[1.5pt]
 测试程序 & \multicolumn{1}{c}{正常运行} & \multicolumn{1}{c}{同步} & \multicolumn{1}{c}{检查点} & \multicolumn{1}{c}{卷回恢复}
& \multicolumn{1}{c}{进程迁移} & \multicolumn{1}{c}{检查点} \\
& \multicolumn{1}{c}{时间 (s)}& \multicolumn{1}{c}{时间 (s)}&
\multicolumn{1}{c}{时间 (s)}& \multicolumn{1}{c}{时间 (s)}& \multicolumn{1}{c}{
  时间 (s)}&  文件(KB)\\\midrule[1pt]
\endhead
\hline
\multicolumn{7}{r}{续下页}
\endfoot
\endlastfoot
CG.A.2 & 23.05 & 0.002 & 0.116 & 0.035 & 0.589 & 32491 \\
CG.A.4 & 15.06 & 0.003 & 0.067 & 0.021 & 0.351 & 18211 \\
CG.A.8 & 13.38 & 0.004 & 0.072 & 0.023 & 0.210 & 9890 \\
CG.B.2 & 867.45 & 0.002 & 0.864 & 0.232 & 3.256 & 228562 \\
CG.B.4 & 501.61 & 0.003 & 0.438 & 0.136 & 2.075 & 123862 \\
CG.B.8 & 384.65 & 0.004 & 0.457 & 0.108 & 1.235 & 63777 \\
MG.A.2 & 112.27 & 0.002 & 0.846 & 0.237 & 3.930 & 236473 \\
MG.A.4 & 59.84 & 0.003 & 0.442 & 0.128 & 2.070 & 123875 \\
MG.A.8 & 31.38 & 0.003 & 0.476 & 0.114 & 1.041 & 60627 \\
MG.B.2 & 526.28 & 0.002 & 0.821 & 0.238 & 4.176 & 236635 \\
MG.B.4 & 280.11 & 0.003 & 0.432 & 0.130 & 1.706 & 123793 \\
MG.B.8 & 148.29 & 0.003 & 0.442 & 0.116 & 0.893 & 60600 \\
LU.A.2 & 2116.54 & 0.002 & 0.110 & 0.030 & 0.532 & 28754 \\
LU.A.4 & 1102.50 & 0.002 & 0.069 & 0.017 & 0.255 & 14915 \\
LU.A.8 & 574.47 & 0.003 & 0.067 & 0.016 & 0.192 & 8655 \\
LU.B.2 & 9712.87 & 0.002 & 0.357 & 0.104 & 1.734 & 101975 \\
LU.B.4 & 4757.80 & 0.003 & 0.190 & 0.056 & 0.808 & 53522 \\
LU.B.8 & 2444.05 & 0.004 & 0.222 & 0.057 & 0.548 & 30134 \\
EP.A.2 & 123.81 & 0.002 & 0.010 & 0.003 & 0.074 & 1834 \\
EP.A.4 & 61.92 & 0.003 & 0.011 & 0.004 & 0.073 & 1743 \\
EP.A.8 & 31.06 & 0.004 & 0.017 & 0.005 & 0.073 & 1661 \\
EP.B.2 & 495.49 & 0.001 & 0.009 & 0.003 & 0.196 & 2011 \\
EP.B.4 & 247.69 & 0.002 & 0.012 & 0.004 & 0.122 & 1663 \\
EP.B.8 & 126.74 & 0.003 & 0.017 & 0.005 & 0.083 & 1656 \\
\bottomrule[1.5pt]
\end{longtable}

\subsection{其它}
\label{sec:tableother}
如果不想让某个表格或者图片出现在索引里面,请使用命令 \verb|\caption*|。
这个命令不会给表格编号,也就是出来的只有标题文字而没有“表~XX”,“图~XX”,否则
索引里面序号不连续就显得不伦不类,这也是 \LaTeX{} 里星号命令默认的规则。

有这种需求的多是本科同学的英文资料翻译部分,如果觉得附录中英文原文中的表格和图
片显示成“表”和“图”不协调的话,一个很好的办法就是用 \verb|\caption*|,参数
随便自己写,比如不守规矩的表~1.111 和图~1.111 能满足这种特殊需要(可以参看附录部
分)。
\begin{table}[ht]
  \begin{minipage}{0.4\linewidth}
    \centering
    \caption*{表~1.111\quad 这是一个手动编号,不出现在索引中的表格。}
    \label{tab:badtabular}
      \framebox(150,50)[c]{\bnuthesis}
  \end{minipage}%
  \hfill%
  \begin{minipage}{0.4\linewidth}
    \centering
    \caption*{Figure~1.111\quad 这是一个手动编号,不出现在索引中的图。}
    \label{tab:badfigure}
    \framebox(150,50)[c]{薛瑞尼}
  \end{minipage}
\end{table}

如果的确想让它编号,但又不想让它出现在索引中的话,目前模板上不支持。

最后,虽然大家不一定会独立使用小页,但是关于小页中的脚注还是有必要提一下。请看下
面的例子。

\begin{minipage}[t]{\linewidth-2\parindent}
  柳宗元,字子厚(773-819),河东(今永济县)人\footnote{山西永济水饺。},是唐代
  杰出的文学家,哲学家,同时也是一位政治改革家。与韩愈共同倡导唐代古文运动,并称
  韩柳\footnote{唐宋八大家之首二位。}。
\end{minipage}

\section{定理环境}
\label{sec:theorem}

给大家演示一下各种和证明有关的环境:

\begin{assumption}
待月西厢下,迎风户半开;隔墙花影动,疑是玉人来。
\begin{eqnarray}
  \label{eq:eqnxmp}
  c & = & a^2 - b^2\\
    & = & (a+b)(a-b)
\end{eqnarray}
\end{assumption}

\begin{definition}
子曰:「道千乘之国,敬事而信,节用而爱人,使民以时。」
\end{definition}

\begin{proposition}
 曾子曰:「吾日三省吾身 —— 为人谋而不忠乎?与朋友交而不信乎?传不习乎?」
\end{proposition}

\begin{remark}
天不言自高,水不言自流。
\begin{gather*}
\begin{split} 
\varphi(x,z)
&=z-\gamma_{10}x-\gamma_{mn}x^mz^n\\
&=z-Mr^{-1}x-Mr^{-(m+n)}x^mz^n
\end{split}\\[6pt]
\begin{align} \zeta^0&=(\xi^0)^2,\\
\zeta^1 &=\xi^0\xi^1,\\
\zeta^2 &=(\xi^1)^2,
\end{align}
\end{gather*}
\end{remark}

\begin{axiom}
两点间直线段距离最短。  
\begin{align}
x&\equiv y+1\pmod{m^2}\\
x&\equiv y+1\mod{m^2}\\
x&\equiv y+1\pod{m^2}
\end{align}
\end{axiom}

\begin{lemma}
《猫和老鼠》是我最爱看的动画片。
\begin{multline*}%\tag*{[a]} % 这个不出现在索引中
\int_a^b\biggl\{\int_a^b[f(x)^2g(y)^2+f(y)^2g(x)^2]
 -2f(x)g(x)f(y)g(y)\,dx\biggr\}\,dy \\
 =\int_a^b\biggl\{g(y)^2\int_a^bf^2+f(y)^2
  \int_a^b g^2-2f(y)g(y)\int_a^b fg\biggr\}\,dy
\end{multline*}
\end{lemma}

\begin{theorem}\label{the:theorem1}
犯我强汉者,虽远必诛\hfill —— 陈汤(汉)
\end{theorem}
\begin{subequations}
\begin{align}
y & = 1 \\
y & = 0
\end{align}
\end{subequations}


\begin{proof}
燕赵古称多感慨悲歌之士。董生举进士,连不得志于有司,怀抱利器,郁郁适兹土,吾
知其必有合也。董生勉乎哉?

夫以子之不遇时,苟慕义强仁者,皆爱惜焉,矧燕、赵之士出乎其性者哉!然吾尝闻
风俗与化移易,吾恶知其今不异于古所云邪?聊以吾子之行卜之也。董生勉乎哉?

吾因子有所感矣。为我吊望诸君之墓,而观于其市,复有昔时屠狗者乎?为我谢
曰:“明天子在上,可以出而仕矣!” \hfill —— 韩愈《送董邵南序》
\end{proof}

\begin{corollary}
  四川话配音的《猫和老鼠》是世界上最好看最好听最有趣的动画片。
\begin{alignat}{3}
V_i & =v_i - q_i v_j, & \qquad X_i & = x_i - q_i x_j,
 & \qquad U_i & = u_i,
 \qquad \text{for $i\ne j$;}\label{eq:B}\\
V_j & = v_j, & \qquad X_j & = x_j,
  & \qquad U_j & u_j + \sum_{i\ne j} q_i u_i.
\end{alignat}
\end{corollary}

迢迢牵牛星,皎皎河汉女。
纤纤擢素手,札札弄机杼。
终日不成章,泣涕零如雨。
河汉清且浅,相去复几许。
盈盈一水间,脉脉不得语。

\begin{example}
  大家来看这个例子。
\begin{equation}
\label{ktc}
\left\{\begin{array}{l}
\nabla f({\mbox{\boldmath $x$}}^*)-\sum\limits_{j=1}^p\lambda_j\nabla g_j({\mbox{\boldmath $x$}}^*)=0\\[0.3cm]
\lambda_jg_j({\mbox{\boldmath $x$}}^*)=0,\quad j=1,2,\cdots,p\\[0.2cm]
\lambda_j\ge 0,\quad j=1,2,\cdots,p.
\end{array}\right.
\end{equation}
\end{example}

\begin{exercise}
  清列出 Andrew S. Tanenbaum 和 W. Richard Stevens 的所有著作。
\end{exercise}

\begin{conjecture} \textit{Poincare Conjecture} If in a closed three-dimensional
  space, any closed curves can shrink to a point continuously, this space can be
  deformed to a sphere.
\end{conjecture}

\begin{problem}
 回答还是不回答,是个问题。 
\end{problem}

如何引用定理~\ref{the:theorem1} 呢?加上 \verb|label| 使用 \verb|ref| 即可。

\section{参考文献}
\label{sec:bib}
当然参考文献可以直接写 \verb|bibitem|,虽然费点功夫,但是好控制,各种格式可以自己随意改
写。

本模板推荐使用 BIB\TeX,样式文件为 \texttt{bnutils.bst},基本符合学校的参考文献格
式(如专利等引用未加详细测试)。看看这个例子,关于书的~\cite{tex, companion,
  ColdSources},还有这些~\cite{Krasnogor2004e, clzs, zjsw},关于杂志
的~\cite{ELIDRISSI94, MELLINGER96, SHELL02},硕士论文~\cite{zhubajie,
  metamori2004},博士论文~\cite{shaheshang, FistSystem01},标准文
件~\cite{IEEE-1363},会议论文~\cite{DPMG,kocher99},技术报告~\cite{NPB2},电子文
献~\cite{chuban2001,oclc2000}。中文参考文献~\cite{cnarticle}应增
加 \texttt{lang=``zh''} 字段,以便进行相应处理。另外,本模板对中文文
献~\cite{cnproceed}的支持并不是十全十美,如果有不如意的地方,请手动修
改 \texttt{bbl} 文件。

有时候不想要上标,那么可以这样~\onlinecite{shaheshang},这个非常重要。

有时候一些参考文献没有纸质出处,需要标注 URL。缺省情况下,URL 不会在连字符处断行,
这可能使得用连字符代替空格的网址分行很难看。如果需要,可以将模板类文件中
\begin{verbatim}
\RequirePackage{hyperref}
\end{verbatim}
一行改为:
\begin{verbatim}
\PassOptionsToPackage{hyphens}{url}
\RequirePackage{hyperref}
\end{verbatim}
使得连字符处可以断行。更多设置可以参考 \texttt{url} 宏包文档。

\section{公式}
\label{sec:equation}
贝叶斯公式如式~(\ref{equ:chap1:bayes}),其中 $p(y|\mathbf{x})$ 为后验;
$p(\mathbf{x})$ 为先验;分母 $p(\mathbf{x})$ 为归一化因子。
\begin{equation}
\label{equ:chap1:bayes}
p(y|\mathbf{x}) = \frac{p(\mathbf{x},y)}{p(\mathbf{x})}=
\frac{p(\mathbf{x}|y)p(y)}{p(\mathbf{x})} 
\end{equation}

论文里面公式越多,\TeX{} 就越 happy。再看一个 \textsf{amsmath} 的例子:
\newcommand{\envert}[1]{\left\lvert#1\right\rvert} 
\begin{equation}\label{detK2}
\det\mathbf{K}(t=1,t_1,\dots,t_n)=\sum_{I\in\mathbf{n}}(-1)^{\envert{I}}
\prod_{i\in I}t_i\prod_{j\in I}(D_j+\lambda_jt_j)\det\mathbf{A}
^{(\lambda)}(\overline{I}|\overline{I})=0.
\end{equation} 

前面定理示例部分列举了很多公式环境,可以说把常见的情况都覆盖了,大家在写公式的时
候一定要好好看 \textsf{amsmath} 的文档,并参考模板中的用法:
\begin{multline*}%\tag{[b]} % 这个出现在索引中的
\int_a^b\biggl\{\int_a^b[f(x)^2g(y)^2+f(y)^2g(x)^2]
 -2f(x)g(x)f(y)g(y)\,dx\biggr\}\,dy \\
 =\int_a^b\biggl\{g(y)^2\int_a^bf^2+f(y)^2
  \int_a^b g^2-2f(y)g(y)\int_a^b fg\biggr\}\,dy
\end{multline*}

其实还可以看看这个多级规划:
\begin{equation}\label{bilevel}
\left\{\begin{array}{l}
\max\limits_{{\mbox{\footnotesize\boldmath $x$}}} F(x,y_1^*,y_2^*,\cdots,y_m^*)\\[0.2cm]
\mbox{subject to:}\\[0.1cm]
\qquad G(x)\le 0\\[0.1cm]
\qquad(y_1^*,y_2^*,\cdots,y_m^*)\mbox{ solves problems }(i=1,2,\cdots,m)\\[0.1cm]
\qquad\left\{\begin{array}{l}
    \max\limits_{{\mbox{\footnotesize\boldmath $y_i$}}}f_i(x,y_1,y_2,\cdots,y_m)\\[0.2cm]
    \mbox{subject to:}\\[0.1cm]
    \qquad g_i(x,y_1,y_2,\cdots,y_m)\le 0.
    \end{array}\right.
\end{array}\right.
\end{equation}
这些跟规划相关的公式都来自于刘宝碇老师《不确定规划》的课件。

\chapter{中华人民共和国}
\label{cha:china}

\section{其它例子}
\label{sec:other}

在第~\ref{cha:intro} 章中我们学习了贝叶斯公式~(\ref{equ:chap1:bayes}),这里我们复
习一下:
\begin{equation}
\label{equ:chap2:bayes}
p(y|\mathbf{x}) = \frac{p(\mathbf{x},y)}{p(\mathbf{x})}=
\frac{p(\mathbf{x}|y)p(y)}{p(\mathbf{x})}
\end{equation}

\subsection{绘图}
\label{sec:draw}

本模板不再预先装载任何绘图包(如 \textsf{pstricks,pgf} 等),完全由用户来决定。
个人觉得 \textsf{pgf} 不错,不依赖于 Postscript。此外还有很多针对 \LaTeX{} 的
 GUI 作图工具,如 XFig(jFig), WinFig, Tpx, Ipe, Dia, Inkscape, LaTeXPiX,
jPicEdt, jaxdraw 等等。

\subsection{插图}
\label{sec:graphs}

强烈推荐《\LaTeXe\ 插图指南》!关于子图形的使用细节请参看 \textsf{subcaption} 宏包的说明文档。

\subsubsection{一个图形}
\label{sec:onefig}
一般图形都是处在浮动环境中。之所以称为浮动是指最终排版效果图形的位置不一定与源文
件中的位置对应\footnote{This is not a bug, but a feature of \LaTeX!},这也是刚使
用 \LaTeX{} 同学可能遇到的问题。如果要强制固定浮动图形的位置,请使用 \textsf{float} 宏包,
它提供了 \texttt{[H]} 参数,比如图~\ref{fig:xfig1}。
\begin{figure}[H] % use float package if you want it here
  \centering
  \includegraphics{thu-whole-logo}
  \caption{利用 Xfig 制图}
  \label{fig:xfig1}
\end{figure}

大学之道,在明明德,在亲民,在止于至善。知止而后有定;定而后能静;静而后能安;安
而后能虑;虑而后能得。物有本末,事有终始。知所先后,则近道矣。古之欲明明德于天
下者,先治其国;欲治其国者,先齐其家;欲齐其家者,先修其身;欲修其身者,先正其心;
欲正其心者,先诚其意;欲诚其意者,先致其知;致知在格物。物格而后知至;知至而后
意诚;意诚而后心正;心正而后身 修;身修而后家齐;家齐而后国治;国治而后天下
平。自天子以至于庶人,壹是皆以修身为本。其本乱而未治者 否矣。其所厚者薄,而其所
薄者厚,未之有也!

\hfill —— 《大学》


\subsubsection{多个图形}
\label{sec:multifig}

%如果多个图形相互独立,并不共用一个图形计数器,那么
%用 \texttt{minipage} 或者\texttt{parbox} 就可以。否则,请参看
%图~\ref{fig:big1-subfloat},它包含两个小图,分别是图~\ref{fig:subfig1}和
%图~\ref{fig:subfig2}。推荐使用\verb|subfloat|,因为可以像
%图~\ref{fig:big1-subfloat} 那样对齐子图的标题,也可以使
%用\textsf{subcaption}宏包的\verb|subcaption|(放在 minipage中,用法同\verb|caption|)。

%\begin{figure}[h]
  %\centering%
  %\subfloat{第一个小图形\label{fig:subfig1}}[3cm] %标题的长度,超过则会换行,如下一个小图。
    %{\includegraphics[height=3cm]{thu-fig-logo}}%
  %\hspace{4em}%
  %\subfloat{第二个小图形,注意这个图略矮些。如果标题很长的话,它会自动换行\label{fig:subfig2}}
      %{\includegraphics[height=2cm]{thu-text-logo}}
  %\caption{包含子图形的大图形(subfloat示例)}
  %\label{fig:big1-subfloat}
%\end{figure}
%\begin{figure}[h]
  %\centering%
  %\begin{subfigure}{3cm}
    %\includegraphics[height=3cm]{thu-fig-logo}
    %\caption{第一个小图形}
  %\end{subfigure}%
  %\hspace{4em}%
  %\begin{subfigure}{0.5\textwidth}
    %\includegraphics[height=2cm]{thu-text-logo}
    %\caption{第二个小图形,注意这个图略矮些。subfigure中同一行的子图在顶端对齐。}
  %\end{subfigure}
  %\caption{包含子图形的大图形(subfigure示例)}
  %\label{fig:big1-subfigure}
%\end{figure}

如果要把编号的两个图形并排,那么小页就非常有用了:
\begin{figure}
\begin{minipage}{0.48\textwidth}
  \centering
  \includegraphics[height=2cm]{thu-whole-logo}
  \caption{并排第一个图}
  \label{fig:parallel1}
\end{minipage}\hfill
\begin{minipage}{0.48\textwidth}
  \centering
  \includegraphics[height=2cm]{thu-whole-logo}
  \caption{并排第二个图}
  \label{fig:parallel2}
\end{minipage}
\end{figure}



%%% 其它部分
\backmatter

%% 本科生要这几个索引,研究生不要。选择性留下。
% 插图索引
\listoffigures
% 表格索引
\listoftables
% 公式索引
\listofequations


%% 参考文献
% 注意:至少需要引用一篇参考文献,否则下面两行可能引起编译错误。
% 如果不需要参考文献,请将下面两行删除或注释掉。
\bibliographystyle{bnubib}
\bibliography{ref/refs}


%% 致谢
% 如果使用声明扫描页,将可选参数指定为扫描后的 PDF 文件名,例如:
% \begin{acknowledgement}[scan-statement.pdf]
\begin{ack}
  衷心感谢导师 xxx 教授和物理系 xxx 副教授对本人的精心指导。他们的言传身教将使
  我终生受益。

  在美国麻省理工学院化学系进行九个月的合作研究期间,承蒙 xxx 教授热心指导与帮助,不
  胜感激。感谢 xx 实验室主任 xx 教授,以及实验室全体老师和同学们的热情帮助和支
  持!本课题承蒙国家自然科学基金资助,特此致谢。

  感谢 \bnuthesis,它的存在让我的论文写作轻松自在了许多,让我的论文格式规整漂亮了
  许多。
\end{ack}


%% 附录
\begin{appendix}
\chapter{外文资料原文}
\label{cha:engorg}

\title{The title of the English paper}

\textbf{Abstract:} As one of the most widely used techniques in operations
research, \emph{ mathematical programming} is defined as a means of maximizing a
quantity known as \emph{bjective function}, subject to a set of constraints
represented by equations and inequalities. Some known subtopics of mathematical
programming are linear programming, nonlinear programming, multiobjective
programming, goal programming, dynamic programming, and multilevel
programming$^{[1]}$.

It is impossible to cover in a single chapter every concept of mathematical
programming. This chapter introduces only the basic concepts and techniques of
mathematical programming such that readers gain an understanding of them
throughout the book$^{[2,3]}$.


\section{Single-Objective Programming}
The general form of single-objective programming (SOP) is written
as follows,
\begin{equation}\tag*{(123)} % 如果附录中的公式不想让它出现在公式索引中,那就请
                             % 用 \tag*{xxxx}
\left\{\begin{array}{l}
\max \,\,f(x)\\[0.1 cm]
\mbox{subject to:} \\ [0.1 cm]
\qquad g_j(x)\le 0,\quad j=1,2,\cdots,p
\end{array}\right.
\end{equation}
which maximizes a real-valued function $f$ of
$x=(x_1,x_2,\cdots,x_n)$ subject to a set of constraints.

\newtheorem{mpdef}{Definition}[chapter]
\begin{mpdef}
In SOP, we call $x$ a decision vector, and
$x_1,x_2,\cdots,x_n$ decision variables. The function
$f$ is called the objective function. The set
\begin{equation}\tag*{(456)} % 这里同理,其它不再一一指定。
S=\left\{x\in\Re^n\bigm|g_j(x)\le 0,\,j=1,2,\cdots,p\right\}
\end{equation}
is called the feasible set. An element $x$ in $S$ is called a
feasible solution.
\end{mpdef}

\newtheorem{mpdefop}[mpdef]{Definition}
\begin{mpdefop}
A feasible solution $x^*$ is called the optimal
solution of SOP if and only if
\begin{equation}
f(x^*)\ge f(x)
\end{equation}
for any feasible solution $x$.
\end{mpdefop}

One of the outstanding contributions to mathematical programming was known as
the Kuhn-Tucker conditions\ref{eq:ktc}. In order to introduce them, let us give
some definitions. An inequality constraint $g_j(x)\le 0$ is said to be active at
a point $x^*$ if $g_j(x^*)=0$. A point $x^*$ satisfying $g_j(x^*)\le 0$ is said
to be regular if the gradient vectors $\nabla g_j(x)$ of all active constraints
are linearly independent.

Let $x^*$ be a regular point of the constraints of SOP and assume that all the
functions $f(x)$ and $g_j(x),j=1,2,\cdots,p$ are differentiable. If $x^*$ is a
local optimal solution, then there exist Lagrange multipliers
$\lambda_j,j=1,2,\cdots,p$ such that the following Kuhn-Tucker conditions hold,
\begin{equation}
\label{eq:ktc}
\left\{\begin{array}{l}
    \nabla f(x^*)-\sum\limits_{j=1}^p\lambda_j\nabla g_j(x^*)=0\\[0.3cm]
    \lambda_jg_j(x^*)=0,\quad j=1,2,\cdots,p\\[0.2cm]
    \lambda_j\ge 0,\quad j=1,2,\cdots,p.
\end{array}\right.
\end{equation}
If all the functions $f(x)$ and $g_j(x),j=1,2,\cdots,p$ are convex and
differentiable, and the point $x^*$ satisfies the Kuhn-Tucker conditions
(\ref{eq:ktc}), then it has been proved that the point $x^*$ is a global optimal
solution of SOP.

\subsection{Linear Programming}
\label{sec:lp}

If the functions $f(x),g_j(x),j=1,2,\cdots,p$ are all linear, then SOP is called
a {\em linear programming}.

The feasible set of linear is always convex. A point $x$ is called an extreme
point of convex set $S$ if $x\in S$ and $x$ cannot be expressed as a convex
combination of two points in $S$. It has been shown that the optimal solution to
linear programming corresponds to an extreme point of its feasible set provided
that the feasible set $S$ is bounded. This fact is the basis of the {\em simplex
  algorithm} which was developed by Dantzig as a very efficient method for
solving linear programming.
\begin{table}[ht]
\centering
  \centering
  \caption*{Table~1\hskip1em This is an example for manually numbered table, which
    would not appear in the list of tables}
  \label{tab:badtabular2}
  \begin{tabular}[c]{|m{1.5cm}|c|c|c|c|c|c|}\hline
    \multicolumn{2}{|c|}{Network Topology} & \# of nodes &
    \multicolumn{3}{c|}{\# of clients} & Server \\\hline
    GT-ITM & Waxman Transit-Stub & 600 &
    \multirow{2}{2em}{2\%}&
    \multirow{2}{2em}{10\%}&
    \multirow{2}{2em}{50\%}&
    \multirow{2}{1.2in}{Max. Connectivity}\\\cline{1-3}
    \multicolumn{2}{|c|}{Inet-2.1} & 6000 & & & &\\\hline
    \multirow{2}{1.5cm}{Xue} & Rui  & Ni &\multicolumn{4}{c|}{\multirow{2}*{\bnuthesis}}\\\cline{2-3}
    & \multicolumn{2}{c|}{ABCDEF} &\multicolumn{4}{c|}{} \\\hline
\end{tabular}
\end{table}

Roughly speaking, the simplex algorithm examines only the extreme points of the
feasible set, rather than all feasible points. At first, the simplex algorithm
selects an extreme point as the initial point. The successive extreme point is
selected so as to improve the objective function value. The procedure is
repeated until no improvement in objective function value can be made. The last
extreme point is the optimal solution.

\subsection{Nonlinear Programming}

If at least one of the functions $f(x),g_j(x),j=1,2,\cdots,p$ is nonlinear, then
SOP is called a {\em nonlinear programming}.

A large number of classical optimization methods have been developed to treat
special-structural nonlinear programming based on the mathematical theory
concerned with analyzing the structure of problems.
\begin{figure}[h]
  \centering
  \includegraphics{thu-lib-logo}
  \caption*{Figure~1\quad This is an example for manually numbered figure,
    which would not appear in the list of figures}
  \label{tab:badfigure2}
\end{figure}

Now we consider a nonlinear programming which is confronted solely with
maximizing a real-valued function with domain $\Re^n$.  Whether derivatives are
available or not, the usual strategy is first to select a point in $\Re^n$ which
is thought to be the most likely place where the maximum exists. If there is no
information available on which to base such a selection, a point is chosen at
random. From this first point an attempt is made to construct a sequence of
points, each of which yields an improved objective function value over its
predecessor. The next point to be added to the sequence is chosen by analyzing
the behavior of the function at the previous points. This construction continues
until some termination criterion is met. Methods based upon this strategy are
called {\em ascent methods}, which can be classified as {\em direct methods},
{\em gradient methods}, and {\em Hessian methods} according to the information
about the behavior of objective function $f$. Direct methods require only that
the function can be evaluated at each point. Gradient methods require the
evaluation of first derivatives of $f$. Hessian methods require the evaluation
of second derivatives. In fact, there is no superior method for all
problems. The efficiency of a method is very much dependent upon the objective
function.

\subsection{Integer Programming}

{\em Integer programming} is a special mathematical programming in which all of
the variables are assumed to be only integer values. When there are not only
integer variables but also conventional continuous variables, we call it {\em
  mixed integer programming}. If all the variables are assumed either 0 or 1,
then the problem is termed a {\em zero-one programming}. Although integer
programming can be solved by an {\em exhaustive enumeration} theoretically, it
is impractical to solve realistically sized integer programming problems. The
most successful algorithm so far found to solve integer programming is called
the {\em branch-and-bound enumeration} developed by Balas (1965) and Dakin
(1965). The other technique to integer programming is the {\em cutting plane
  method} developed by Gomory (1959).

\hfill\textit{Uncertain Programming\/}\quad(\textsl{BaoDing Liu, 2006.2})

\section*{References}
\noindent{\itshape NOTE: These references are only for demonstration. They are
  not real citations in the original text.}

%\begin{translationbib}
%\item Donald E. Knuth. The \TeX book. Addison-Wesley, 1984. ISBN: 0-201-13448-9
%\item Paul W. Abrahams, Karl Berry and Kathryn A. Hargreaves. \TeX\ for the
  %Impatient. Addison-Wesley, 1990. ISBN: 0-201-51375-7
%\item David Salomon. The advanced \TeX book.  New York : Springer, 1995. ISBN:0-387-94556-3
%\end{translationbib}

\chapter{外文资料的调研阅读报告或书面翻译}

\title{英文资料的中文标题}

{\heiti 摘要:} 本章为外文资料翻译内容。如果有摘要可以直接写上来,这部分好像没有
明确的规定。

\section{单目标规划}
北冥有鱼,其名为鲲。鲲之大,不知其几千里也。化而为鸟,其名为鹏。鹏之背,不知其几
千里也。怒而飞,其翼若垂天之云。是鸟也,海运则将徙于南冥。南冥者,天池也。
\begin{equation}\tag*{(123)}
 p(y|\mathbf{x}) = \frac{p(\mathbf{x},y)}{p(\mathbf{x})}=
\frac{p(\mathbf{x}|y)p(y)}{p(\mathbf{x})}
\end{equation}

吾生也有涯,而知也无涯。以有涯随无涯,殆已!已而为知者,殆而已矣!为善无近名,为
恶无近刑,缘督以为经,可以保身,可以全生,可以养亲,可以尽年。

\subsection{线性规划}
庖丁为文惠君解牛,手之所触,肩之所倚,足之所履,膝之所倚,砉然响然,奏刀騞然,莫
不中音,合于桑林之舞,乃中经首之会。
\begin{table}[ht]
\centering
  \centering
  \caption*{表~1\hskip1em 这是手动编号但不出现在索引中的一个表格例子}
  \label{tab:badtabular3}
  \begin{tabular}[c]{|m{1.5cm}|c|c|c|c|c|c|}\hline
    \multicolumn{2}{|c|}{Network Topology} & \# of nodes &
    \multicolumn{3}{c|}{\# of clients} & Server \\\hline
    GT-ITM & Waxman Transit-Stub & 600 &
    \multirow{2}{2em}{2\%}&
    \multirow{2}{2em}{10\%}&
    \multirow{2}{2em}{50\%}&
    \multirow{2}{1.2in}{Max. Connectivity}\\\cline{1-3}
    \multicolumn{2}{|c|}{Inet-2.1} & 6000 & & & &\\\hline
    \multirow{2}{1.5cm}{Xue} & Rui  & Ni &\multicolumn{4}{c|}{\multirow{2}*{\bnuthesis}}\\\cline{2-3}
    & \multicolumn{2}{c|}{ABCDEF} &\multicolumn{4}{c|}{} \\\hline
\end{tabular}
\end{table}

文惠君曰:“嘻,善哉!技盖至此乎?”庖丁释刀对曰:“臣之所好者道也,进乎技矣。始臣之
解牛之时,所见无非全牛者;三年之后,未尝见全牛也;方今之时,臣以神遇而不以目视,
官知止而神欲行。依乎天理,批大郤,导大窾,因其固然。技经肯綮之未尝,而况大坬乎!
良庖岁更刀,割也;族庖月更刀,折也;今臣之刀十九年矣,所解数千牛矣,而刀刃若新发
于硎。彼节者有间而刀刃者无厚,以无厚入有间,恢恢乎其于游刃必有余地矣。是以十九年
而刀刃若新发于硎。虽然,每至于族,吾见其难为,怵然为戒,视为止,行为迟,动刀甚微,
謋然已解,如土委地。提刀而立,为之而四顾,为之踌躇满志,善刀而藏之。”

文惠君曰:“善哉!吾闻庖丁之言,得养生焉。”


\subsection{非线性规划}
孔子与柳下季为友,柳下季之弟名曰盗跖。盗跖从卒九千人,横行天下,侵暴诸侯。穴室枢
户,驱人牛马,取人妇女。贪得忘亲,不顾父母兄弟,不祭先祖。所过之邑,大国守城,小
国入保,万民苦之。孔子谓柳下季曰:“夫为人父者,必能诏其子;为人兄者,必能教其弟。
若父不能诏其子,兄不能教其弟,则无贵父子兄弟之亲矣。今先生,世之才士也,弟为盗
跖,为天下害,而弗能教也,丘窃为先生羞之。丘请为先生往说之。”
\begin{figure}[h]
  \centering
  \includegraphics{thu-whole-logo}
  \caption*{图~1\hskip1em 这是手动编号但不出现索引中的图片的例子}
  \label{tab:badfigure3}
\end{figure}

柳下季曰:“先生言为人父者必能诏其子,为人兄者必能教其弟,若子不听父之诏,弟不受
兄之教,虽今先生之辩,将奈之何哉?且跖之为人也,心如涌泉,意如飘风,强足以距敌,
辩足以饰非。顺其心则喜,逆其心则怒,易辱人以言。先生必无往。”

孔子不听,颜回为驭,子贡为右,往见盗跖。

\subsection{整数规划}
盗跖乃方休卒徒大山之阳,脍人肝而餔之。孔子下车而前,见谒者曰:“鲁人孔丘,闻将军
高义,敬再拜谒者。”谒者入通。盗跖闻之大怒,目如明星,发上指冠,曰:“此夫鲁国之
巧伪人孔丘非邪?为我告之:尔作言造语,妄称文、武,冠枝木之冠,带死牛之胁,多辞缪
说,不耕而食,不织而衣,摇唇鼓舌,擅生是非,以迷天下之主,使天下学士不反其本,妄
作孝弟,而侥幸于封侯富贵者也。子之罪大极重,疾走归!不然,我将以子肝益昼餔之膳。”


\chapter{其它附录}
前面两个附录主要是给本科生做例子。其它附录的内容可以放到这里,当然如果你愿意,可
以把这部分也放到独立的文件中,然后将其 \verb|input| 到主文件中。

\end{appendix}

%%学术成果
\include{data/paper}

%% 本科生进行格式审查是需要下面这个表格,答辩可能不需要。选择性留下。
% 综合论文训练记录表
%\includepdf[pages=-]{scan-record.pdf}
\end{document}
